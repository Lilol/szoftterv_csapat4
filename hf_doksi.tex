\documentclass[12pt]{article}\usepackage[left=20mm,right=20mm,top=14mm,bottom=20mm]{geometry}

\usepackage[T1]{fontenc}
\usepackage[magyar]{babel}
\usepackage[utf8]{inputenc}
\usepackage[table,xcdraw]{xcolor}

\usepackage{graphicx}
\usepackage{caption}
\usepackage{siunitx}
\usepackage{amsmath}
\usepackage{epstopdf}
\usepackage{multirow}
\usepackage{makeidx}
\usepackage{placeins}
\usepackage{subcaption}
\usepackage{hyperref}
\usepackage{pdflscape}

%\DeclareUnicodeCharacter{00A0}{~}

\begin{document}
\begin{titlepage}
\centering
	\includegraphics[width=0.5\textwidth]{figures/bme_logo_kicsi.eps}\par\vspace{1cm}
	\vspace{1cm}
	\vspace{1.5cm}
	{\huge\bfseries Szoftvertervezés házi feladat \par}
	\vspace{15cm}
	{\huge\itshape Locskai Norbert, Kovács András, Barancsuk Lilla \par}
	\vfill

% Bottom of the page
	{\large \today\par}
\end{titlepage}

\section{Vízió}

\begin{table}[]
\centering
\label{tab:vizio_fejlec}
\begin{tabular}{|l|l|l|l|}
\hline
\rowcolor[HTML]{EFEFEF} 
\multicolumn{1}{|c|}{\cellcolor[HTML]{EFEFEF}\textbf{Verzió}} & \multicolumn{1}{c|}{\cellcolor[HTML]{EFEFEF}\textbf{Dátum}} & \multicolumn{1}{c|}{\cellcolor[HTML]{EFEFEF}\textbf{Leírás}} & \multicolumn{1}{c|}{\cellcolor[HTML]{EFEFEF}\textbf{Készítő}}             \\ \hline
Első verzió                                                   & \today                                                      & Vízió kész                                                   & \begin{tabular}[c]{@{}l@{}}Locskai Norbert\\ Barancsuk Lilla\end{tabular} \\ \hline
                                                              &                                                             &                                                              &                                                                           \\ \hline
\end{tabular}
\end{table}

\subsection{Bevezetés}
A Service4U cég berendezések szervizelésével foglalkozik. Az Device4U nemzetközi cég által
gyártott berendezések szervizelését végzi Magyarországon.
A Service4U vezetői egy olyan számítógépes rendszert szeretnének, mely segítségével képesek
lesznek a raktárban tárolt alkatrész készletet nyomon követni, a rendelés, felhasználás, szállítás, illetve
leltározás folyamatait támogatni.

\subsubsection{Megoldandó probléma}
A programnak nyilván kell tartania a raktárban található összes alkatrészt, illetve azokat az időpontokat, amikor a raktár tartalma változott, a változást, illetve azt, aki a változást bevitte a rendszerbe, valamint segítenie kell a rendelések összeállítását.

Emiatt a rendszernek nyilván kell tartania a cég azon dolgozóit is, akik a programot használják (a vezetőség tagjait, illetve a raktárosokat, akik a rendeléseket átveszik, és az eszközöket kiadják).

Nyilván kell tartania még a szervizelt berendezések típusait, és azok alkatrészeit, valamint képesnek kell lennie frissíteni a berendezések listáját.

A program segítségével a cég vezetőinek legyen lehetősége rendelések összeállítására, illetve azok exportálására többféle formátumban (pdf, xls, txt).
 
Mivel a Service4U minden alkatrészből legalább két darabot kíván raktáron tartani, a programnak értesítenie kell a cég vezetőségét, amennyiben bármely alkatrész száma kettő alá csökken.

Az alkatrészeket a Service4U cég szerelői vételezik ki a raktárból egy-egy berendezés javítása
során. 
A szerelők a kivenni kívánt alkatrészeket munkalapon rögzítik, majd azt átadják a raktárosnak. 
A raktárosnak a rendszer segítségével képesnek kell lennie a kivétel nyilvántartására: az időpont, a szerelő, a raktáros azonosítójának, illetve a kivett alkatrészek rögzítésére.

A szállítást mindig azonos szállító cég végzi, akik adott rendelést mindig egy szállítási egységként szállítják.
A program segítségével a raktárosnak rögzítenie kell tudnia a szállítmány tartalmát és bevételezésének időpontját.

Service4U cég vezető félévente leltározást végeznek a raktárban, amikor elkészítik azt a listát, ami a
raktárban levő alkatrészeket tartalmazza.
A rendszernek alkalmasnak kell lennie arra, hogy az adatbázisban lévő alkatrészeket automatikusan összeszámlálja, valamint a kézi leltárral összevethető formában megjelenítse a számlálás eredményét.


\subsection{Érdekeltek köre - stakeholderek}
\textbf{vezető: } A vezető végzi az alkatrészek megrendelését, kezdeményezi a leltározást, továbbá bevihet új berendezéseket az adatbázisba. Célja a raktár kiürülésének megakadályozása, valamint a fent említett munkafolyamatok gyorsítása, automatizálása.
\vspace{4mm}

\noindent\textbf{szerelő: } A szerelő végzi a gépek javítását, e célból alkatrészeket kér ki a raktárból.  Célja, hogy a szükséges alkatrészek rendelkezésre álljanak, valamint hogy a kivétel gyors legyen.
\vspace{4mm}

\noindent\textbf{raktáros: } A raktáros az a személy, aki a szerelőknek kiadja az alkatrészeket, valamint a beszállítóktól átveszi azokat. Emellett a kézi leltározást végzi. Célja, hogy egyszerűen adminisztrálja a raktár tranzakcióit, valamint értesítést kapjon a leltárazás elkezdéséről.
\vspace{4mm}

\noindent\textbf{rendszergazda: } A rendszergazda a rendszer karbantartását végzi, valamint ő felelős új munkatársak nyilvántartásának naprakészen tartásáért. Célja, hogy a rendszer egyszerűen karbantartható, visszaállítható és bővíthető legyen.

\subsection{Rendszer részei}
\noindent\textbf{adatbázis: } Nyilvántartás, ami tartalmazza a kollégákat, a gépeket, azok alkatrészeit, valamint a gépekben található alkatrészeket, valamint a raktári tranzakciókat, ezen kívül az éves leltár eredményét. A rendszeres biztonsági mentések segítségével az adatbázis állapota bármikor visszaállítható. Megkönnyíti és gyorsítja a cégen belüli adminisztrációt, valamint segíti hiba esetén a rendszer visszaállítását.
\vspace{4mm}

\noindent\textbf{grafikus felület (GUI): } A grafikus felhasználói felület megjeleníti a szükséges adatokat, valamint lehetővé teszi új adatok bevitelét a rendszerbe. 
A GUI átlagfelhasználó számára is lehetővé teszi a rendszer használatát, az adatbázisba való adminisztrációt. 
\vspace{4mm}

\noindent\textbf{adatbázis--GUI interfész: } Az adatbázist és grafikus felületet összekötő interfész, ami lekérdezéseket hajt végre az adatbázisban, valamint a lekérdezések eredményét és az ezekből származtatott adatokat továbbítja a grafikus felületnek.
Ez az interfész megkönnyíti az adatbázis kezelését.
\vspace{4mm}

\noindent\textbf{azonosítófelület: } Az azonosítófelület megakadályozza az illetéktelen hozzáférést. Ennek segítségével beállíthatók a jogosultságok, valamint ellenőrizhető, hogy egyes műveleteket ki és mikor végzett.
\vspace{4mm}

\noindent\textbf{lehetőség távoli hozzáférésre: } Távoli elérés lehetőségét biztosító rendszerrész, aminek segítségével a kollégák hálózaton keresztül is kezdeményezhetnek bizonyos műveleteket.
\vspace{4mm}

\subsection{Rendszer korlátai, határai}
\subsubsection{A rendszer feladatainak határai}
A rendszer nem felelős a raktár fizikai feltöltéséért--ürítéséért, leltározásáért. Azt a feladatot a raktáros látja el.
Emellett a program nem képes a kollégák adatainak automatikus importálására a céges nyilvántartásból. 
Ezeket az adatokat a rendszergazda kézzel viszi be, és rendel a személyekhez jogosultságokat.
Ezen kívül nem a rendszer feladata a rendelések elküldése, ezeket csak összeállítja, valamint olvasható formában megjeleníti és exportálja.

\subsubsection{Technológiai korlátok}
A cég számítógépparkja technológiai korlátokat állít a rendszer elé.
Szerverként egyetlen gép áll rendelkezésre, amely csak korlátozott számú kérést tud egyszerre kiszolgálni. 
A kollégák által használt számítógépek elavult, asztali gépek korlátos erőforrásokkal, továbbá programok telepítésére és törlésére csak a rendszergazdának van jogosultsága.

\section{Fogalomszótár}
\noindent\textbf{berendezés: } A cég által szervizelt gép.
\vspace{4mm}

\noindent\textbf{alkatrész: } A gépek építőelemei, amiket raktárban tárol a cég. 
\vspace{4mm}

\noindent\textbf{szerelő: } A gépek szervizelését végzi. Alkatrészeket vesz ki a raktárból.
\vspace{4mm}

\noindent\textbf{vezető: } Az az ember, aki felelős az alkatrészek rendeléséért, illetve a leltározás elrendeléséért.
\vspace{4mm}

\noindent\textbf{raktáros: } Kezeli a raktárat: kiadja a szerelőnek a neki szükséges alkatrészeket, átveszi a szállító által szállított alkatrészeket, valamint elvégzi a manuális leltárazást.
\vspace{4mm}

\noindent\textbf{szállítólevél: } Az a dokumentum, ami tartalmazza az adott szállítmányban található alkatrészek típusait és darabszámát.
\vspace{4mm}

\noindent\textbf{munkalap: } Az a dokumentum, ami tartalmazza a javítandó berendezéshez szükséges alkatrészeket. 
\vspace{4mm}

\noindent\textbf{rendszergazda: } A cég azon alkalmazottja, aki a kezeli a munkatársak jogosultságait, valamint a rendszer karbantartását végzi.
\vspace{4mm}

\noindent\textbf{rendelés: } Az a dokumentum, ami tartalmazza a megrendelni kívánt alkatrészek darabszámát és típusát.
\vspace{4mm}

\noindent\textbf{leltározás: } Az a folyamat, amelynek során a tényleges raktárkészletet összehasonlítják a nyilvántartással.

\noindent\textbf{jelentés: } A leltározás során készült dokumentum, ami összehasonlítható formában tartalmazza a fizikai számlálás és az elektronikus nyilvántartás eredményeit.

\section{Kiegészítő követelmények leírása}

\begin{table}[]
\centering
\label{tab:srs_fejlec}
\begin{tabular}{|l|l|l|l|}
\hline
\rowcolor[HTML]{EFEFEF} 
\multicolumn{1}{|c|}{\cellcolor[HTML]{EFEFEF}\textbf{Verzió}} & \multicolumn{1}{c|}{\cellcolor[HTML]{EFEFEF}\textbf{Dátum}} & \multicolumn{1}{c|}{\cellcolor[HTML]{EFEFEF}\textbf{Leírás}} & \multicolumn{1}{c|}{\cellcolor[HTML]{EFEFEF}\textbf{Készítő}}             \\ \hline
Első verzió                                                   & \today                                                      & Vízió kész                                                   & \begin{tabular}[c]{@{}l@{}}Locskai Norbert\\ Barancsuk Lilla\end{tabular} \\ \hline
                                                              &                                                             &                                                              &                                                                           \\ \hline
\end{tabular}
\end{table}

\subsection{Funkcionális követelmények}
Mivel a rendszer egy raktár nyilvántartását végzi, ezért szükséges, hogy adatbázis mindig tükrözze a tényleges raktárkészletet, megkönnyítse az adminisztrációt és a leltározást.

\subsection{Használhatóság}
Mivel az egyszerű felhasználók nem jártasak az adatbázis-kezelésben, a grafikus felület ergonomikus és intuitív lehetőséget biztosít ennek kezelésére.

\subsection{Megbízhatóság}
A adatbázisban állapotainak visszaállíthatónak kell lenniük, hiba után is. 
Ezért az adatbázisról rendszeresen (minden munkanapon éjfélkor) mentés készül.
Valamint naplózás segítségével a változások nyomon követhetők.

\subsection{Teljesítmény}
Az adatbázisnak képesnek kell lennie nagyszámú adat tárolására. 
Számszerűen:
\begin{itemize}
\item[•] maximum 10.000 féle alkatrész
\item[•] maximum 10.000 darab alkatrész fajtánként
\item[•] maximum 1000 féle gép
\item[•] egy géphez maximum 1000 alkatrész rendelhető
\end{itemize}

A rendszernek kezelnie kell maximum 10 darab egyidejű hozzáférést. 

\subsection{Támogatottság}
Fontos, hogy az adatbázis bővíthető legyen új kollégákkal, valamint új berendezésekkel.
A cég vállalja a szoftver rendszeres karbantartást, ami félévente esedékes ellenőrzést, hibajavítást jelent.
Valamint lehetőség van telefonos ügyfélszolgálat igénybevételére, illetve hibabejelentésre.

\subsection{Technológiai megkötések}
A GUI-nak a következő böngészőkön a megadott verziószámok fölött kell tudni futnia:
\begin{itemize}
\item[•] Chrome 48.0
\item[•] Mozzilla Firefox 44
\item[•] Internet Explorer 9.0
\item[•] Safari 9.0
\end{itemize}

A cég kis teljesítményű, elavult gépei a szoftvernek korlátos erőforrású környezetben is megfelelően kell működnie. 
A jogosultságok korlátozása miatt a grafikus felületnek böngészőben megjeleníthetőnek kell lennie.
Mivel egyetlen szerver áll rendelkezésre, az adatbázisba befutó kéréseket ennek kell elvárható időn belül (1 s) kiszolgálnia.

\subsection{Interfészek}
\noindent\textbf{vonalkódleolvasó--vonalkódfeldolgozó rendszer: } A rendszer kapcsolatban áll egy vonalkód-értelmező szoftverrel, amely beolvassa és értelmezi az alkatrészek vonalkódjait, majd a beolvasott alkatrész azonosítóját továbbküldi az alkatrész-nyilvántartó rendszernek.

\section{Use Case modell}

\begin{figure}[!h]
    \centering
        \includegraphics[width=\textwidth]{kepek/use_case.pdf}
        \caption{A teljes rendszer use case diagramja.}
\end{figure}

\subsection{Aktorok}
\subsubsection{Elsődleges aktorok}
\begin{itemize}
\item[•] \textbf{vezetők: }  A célja az alkatrészek megrendelése, leltározás kezdeményezése, továbbá új berendezések bevitele az adatbázisba, a raktár kiürülésének mgakadályozása. Elvárása a rendszerrel kapcsolatban, hogy a fent említett folyamatokat automatizáltan, gyorsan végrehajtsa.
 
\item[•] \textbf{szerelő: } A szerelő célja, hogy a szükséges alkatrészekhez a lehető leggyorsabban hozzáférjen, illetve ezek mindig rendelkezésre álljanak.

\item[•] \textbf{raktáros: } A raktáros célja a raktár adminisztációjának gyors, megbízható elvégzése. 
Szeretne időben értesítést kapni a leltár elkezdéséről.

\item[•] \textbf{rendszergazda: } Célja, hogy a rendszer egyszerűen karbantartható, visszaállítható és bővíthető legyen.
\end{itemize}

\subsubsection{Támogató aktorok}
\begin{itemize}
\item[•] \textbf{raktáros: } Célja, hogy a szerelőktől vagy a vezetőségtől érkező kéréseket gyorsan kiszolgálja, valamint egyszerűen intézze az alkatrészek átvételét.
Szeretné, hogy fenti munkafolyamatok jól dokumentáltak legyenek.

\item[•] \textbf{alkatrésznyilvántartó rendszer: } Adatokat tárol és szolgáltat. Kiszolgálja a beérkező kéréseket.
\end{itemize}

\section{Legfontosabb use case-ek}
\subsection{Rendelés (\ref{fig:rendeles}.~ábra)}
\noindent\textbf{azonosító: } UC1
\vspace{4mm}

\noindent\textbf{aktorok: } Elsődleges: vezető.  
\vspace{4mm}

\noindent\textbf{érdekeltek: } Vezető: Szeretné a rendszer keretein belül gyorsan összeállítani a szükséges alkatrészek rendelési listáját.
\vspace{4mm}

\noindent\textbf{előfeltételek: } Egy adott típusú alkatrész készleteinek növelésére igény merül fel.
\vspace{4mm}

\noindent\textbf{hatása: } Rendelési fájl generálódik, ami tartalmazza a rendelt alkatrészek listáját.
\vspace{4mm}

\noindent\textbf{alap forgatókönyv: }
\begin{enumerate}
\item Igény merül fel valamilyen típusú alkatrészre.
\item A vezető bejelentkezik.
\item A vezető megnyitja a rendelés-összeállító ablakot.
\item A vezető alkatrészt ad hozzá a rendeléshez, beállítja a rendelni kívánt alkatrész darabszámát.
\textit{A vezető az előző lépést a rendelés teljes összeállításáig ismétli.}
\item A vezető exportálja az elkészült listát valamilyen formátumban (pdf, szöveg, táblázat).
\item A vezető kilép a rendszerből.
\end{enumerate}
\vspace{4mm}

\noindent\textbf{alternatív forgatókönyv: }
\begin{enumerate}
\item Valamelyik típusú alkatrész száma kettő alá csökken.
\item A vezető bejelentkezik.
\item A vezető megnyitja a rendelés-összeállító ablakot.
\item A vezető lekéri a hiányzó alkatrészek listáját.
\item A vezető a listát hozzáadja a rendeléshez.
\item A vezető alkatrészt ad hozzá a rendeléshez, beállítja a rendelni kívánt alkatrész darabszámát. \\
\textit{A vezető az előző lépést a rendelés teljes összeállításáig ismétli.}
\item A vezető exportálja az elkészült listát valamilyen formátumban (pdf, szöveg, táblázat).
\item A vezető kilép a rendszerből.
\end{enumerate}
\vspace{4mm}

\noindent\textbf{különleges követelmények: } A rendelést csak vezetői jogosultsággal rendelkező személy végezheti.
\vspace{4mm}

\noindent\textbf{technológiai variánsok: } \\Az elkészült rendelés exportálható többféle fájlformátumban:
\vspace*{-3mm}
\begin{itemize}
\item[•] xls
\item[•] pdf
\item[•] txt
\end{itemize}

\begin{figure}[!h]
    \centering
        \includegraphics[width=\textwidth]{kepek/rendeles_SD.pdf}
        \caption{A rendelés szekvenciadiagramja. \label{fig:rendeles}}
\end{figure}

\subsection{Leltárazás (\ref{fig:leltar}.~ábra)}
\noindent\textbf{azonosító: } UC2
\vspace{4mm}

\noindent\textbf{aktorok: } Kezdeményező: vezető, támogató: raktáros.
\vspace{4mm}

\noindent\textbf{érdekeltek: }
\vspace*{-3mm}
\begin{itemize}
\item[•] vezető: Célja a leltár automatizált végrehajtása, illetve a fizikai leltározás kezdeményezése. Fontos neki, hogy a folyamat végén egyetlen, áttekinthető dokumentumban láthassa az eredményt.

\item[•] raktáros: Célja, hogy időben értesüljön arról, hogy el kell kezdenie a fizikai leltározást. 
\end{itemize}
\vspace{4mm}

\noindent\textbf{előfeltételek: } A folyamat előfeltétele, hogy az utolsó leltár óta félév eltelt, illetve vezetői jogosultsággal rendelkező felhasználó kezdeményezze.
\vspace{4mm}

\noindent\textbf{hatása: }
A folyamat végén dokumentum készül, ami tartalmazza a fizikai leltár eredményét, illetve az elektronikus adatbázis összehasonlítását áttekinthető formában.
\vspace{4mm}

\noindent\textbf{alap forgatókönyv: }
\begin{enumerate}
\item A vezető bejelentkezik a rendszerbe.
\item A vezető kezdeményezi a leltár elkezdését.
\item A raktárkészletet a rendszer befagyasztja.
\item A rendszer automatikusan összegzi az adatbázisban található alkatrészeket.
\item A rendszer értesítést küld a raktárosnak a leltár elkezdéséről.
\item A raktáros elvégzi a fizika leltározást.
\item A raktáros beviszi a rendszerbe a számlálás eredményét.
\item A rendszer jelentést készít a két eredményből.
\item A vezető kilép a rendszerből.
\end{enumerate}

\begin{figure}[!h]
    \centering
        \includegraphics[width=\textwidth]{kepek/leltarazas_SD.pdf}
        \caption{A leltározás szekvenciadiagramja.\label{fig:leltar}}
\end{figure}

\subsection{Alkatrész kiadása (\ref{fig:kiadas}.~ábra)}
\noindent\textbf{azonosító: } UC3
\vspace{4mm}

\noindent\textbf{aktorok: } Elsődleges: raktáros, támogató: vonalkód-leolvasó rendszer háttér: vezető.
\vspace{4mm}

\noindent\textbf{érdekeltek: }
\vspace*{-3mm}
\begin{itemize}
\item[•] raktáros: Célja, hogy a lehető legkevesebb adminisztrációval ki tudja szolgálni a szerelőt.
\item[•] vezető: Érdekelt abban, hogy a szükségesnél több alkatrész ne kerüljön felhasználásra.
\end{itemize}
\vspace{4mm}

\noindent\textbf{előfeltételek: } A raktáros megkapja a munkalapot a szerelőtől.
\vspace{4mm}

\noindent\textbf{hatása: } A raktáros átadja a megfelelő alkatrészeket a szerelőnek, a kiadás rögzítésre kerül az adatbázisban.
\vspace{4mm}

\noindent\textbf{alap forgatókönyv: }
\begin{enumerate}
\item A raktáros beviszi a munkalapon lévő adatokat a rendszerbe.
\item A rendszer megjeleníti a szükséges alkatrészeket.
\item A raktáros összegyűjti és vonalkód-leolvasóval beolvassa a megjelenített alkatrészeket.
\item A raktáros lezárja a kiadást.
\item A rendszer értesítést küld a vezetőnek, ha a tranzakció hatására valamely alkatrész száma kettő alá csökkenne.
\end{enumerate}
\vspace{4mm}

\noindent\textbf{alternatív forgatókönyv:} \\
Ha adott alkatrészből nem áll rendelkezésre megfelelő mennyiség, akkor a folyamat a következőképpen módosul a 2. ponttól kezdve:
\begin{enumerate}
\item A rendszer kiírja, hogy adott alkatrész hiányzik.
\item A raktáros törli az adott munkalapot.
\end{enumerate}

\begin{figure}[!h]
    \centering
        \includegraphics[width=\textwidth]{kepek/alkatresz_kiadas_SD.pdf}
        \caption{Az alkatrész kiadás szekvenciadiagramja.\label{fig:kiadas}}
\end{figure}

\subsection{User nyilvántartás}
\textbf{azonosító: } UC4 \\
A rendszergazda a rendszerbe belépve beviszi a számára kialakított beviteli felületen az új munkatársak adatait, jogosultságokat rendel hozzájuk, módosításokat végez, vagy törli őket. 
A folyamat eredményeképpen a kollégák adatainak változásai megjelennek a rendszerben.

\subsection{Bejelentkezés}
\textbf{azonosító: } UC5 \\
A rendszerbe bejelentkezhet raktáros, rendszergazda és vezető is.
Bejelentkezéskor meg kell adni az egyedi azonosítót valamint a jelszót.
A bejelentkezés eredményeképpen a felhasználó a jogosultsági szintjének megfelelő műveletek végrehajtására lesz képes.

\subsection{Automatikus értesítés}
\textbf{azonosító: } UC6 \\
Az automatikus értesítést a rendszer küldi a vezetőknek, ha egy adott típusú alkatrész darabszáma a raktárban kettő alá csökken.
A rendszer minden adatbázis-módosításkor ellenőrzi a darabszámokat, és hiány esetén értesítést küld a vezetőknek, akik ezt a rendszerbe belépve megkapják.

\subsection{Új berendezés bevitele}
\textbf{azonosító: } UC7 \\
Ha módosul a szervizelt berendezések listája, a vezetőség a rendszer nyilvántartását belépés után kiegészítheti az új berendezéssel, valamint a hozzá szükséges alkatrészek listájával.

\subsection{Backlog lekérdezése}
\textbf{azonosító: } UC8 \\
A vezetőség ellenőrzés céljából lekérheti bizonyos tranzakciók naplóját. 
A folyamat eredményeképpen a rendszer áttekinthető formában megjeleníti a kívánt időpontban naplózott bejegyzéseket.

\subsection{Megrendelés átvétele}
\textbf{azonosító: } UC9 \\
A raktáros átveszi az érkezett csomagot a szállítólevéllel együtt, beviszi a rendszerbe az új szállítmányt, majd egyenként leolvassa a szállítmányba tartozó alkatrészek vonalkódját.
Ezután elhelyezi őket a raktárban.
A folyamat végén megnő a raktárban és a nyilvántartásban jelen lévő alkatrészek száma.

\subsection{Biztonsági mentés készítése}
\textbf{azonosító: } UC10 \\
Minden munkanapon éjfélkor a rendszer automatikusan biztonsági mentést készít az adatbázis aktuális állapotáról.
Az ennek eredményeképpen létrejövő fájl segítségével a rendszergazda hiba esetén vissza tudja állítani a rendszer korábbi állapotát.

\subsection{Rendszer visszaállítása}
\textbf{azonosító: } UC11 \\
Hiba fellépése esetén a rendszergazda egy biztonsági mentés segítségével a rendszert vissza tudja vinni korábbi állapotába.

\thispagestyle{empty}
\begin{landscape}
\section{Domain modell}
%\newgeometry{left=50mm,right=2mm,top=98.5mm,bottom=10mm}
\begin{figure}[!h]
    \centering
        \includegraphics[width=1.4\textwidth]{kepek/Domain_model.pdf}
        \caption{Domain modell. A domain modellben ábrázoltuk a valóságban lezajló azon főbb folyamatokat, amelyek meghatározóak lesznek a rendszer szempontjából is. Modellünkben a kezdeményezők mind személyek.
A vezetőség tagjai kezdeményezik a leltározást, valamint állítják össze a rendeléseket.
A rendszergazda nyilvántartja a felhasználókat.
A raktáros felelős az alkatrészek tranzakcióiért (kiviteléért, beszállításkor azok raktárba tételéért), valamint ő a fizikai leltározás felelőse is. 
Fontos az is, hogy ha adott típusú alkatrész száma kettő alá csökken, arról a vezetőség tagjai értesüljenek, és a kritikus számú alkatrészek listája alapján tudja a rendelést összeállítani.}
\end{figure}
\end{landscape}

\section{Design modell}
\subsection{Use case-ek megvalósítása}
A fontosabb use case-ek megvalósítását egy--egy szekvenciadiagramon ábrázoltuk, valamint egy egybefüggő design class diagramon szemléltettük a rendszer működését e három legfontosabb use case alapján.

\subsubsection{Rendelés összeállítása}
A rendelést a vezető kezdeményezi, amihez be kell lépnie a  szoftver webes felületén keresztül.
A rendelés alapja a hiányzó alkatrészek listája, amit a céges szerveren tárolt adatbázis szolgáltat. 
A webes felület biztosítja a vezető számára, hogy a hiányzó alkatrészek listájának felhasználásával, annak kiegészítésével összeállítsa a rendelést, majd azt többféle formátumban exportálja a felületen keresztül.

\begin{figure}[!h]
    \centering
        \includegraphics[width=0.8\textwidth]{kepek/Rendeles_szekvencia.pdf}
        \caption{A rendelés-összeállítás szekvencia diagramja.}
\end{figure}

\thispagestyle{empty}
\begin{landscape}
\subsubsection{Alkatrészek kiadása}
Az alkatrészek kiadását a raktáros végzi a szerelő által átadott munkalap alapján. 
A raktáros az átadott munkalapon szereplő alkatrészeket kiveszi a raktárból, majd a vonalkód-leolvasóval leolvassa ezeket. 
A rendszer összeállít egy listát a kivitt alkatrészekből, majd ezeket törli az adatbázisból.
Az esetleges alkatrészhiány kialakulását (amennyiben egy adott típusú alkatrész száma kettő alá csökken) jelzi a vezetőségnek értesítő üzenetben. 

\begin{figure}[!h]
    \centering
        \includegraphics[width=1.4\textwidth]{kepek/kiadas_szekvencia.pdf}
        \caption{Az alkatrészek kiadásának szekvencia diagramja.}
\end{figure}
\end{landscape}

\thispagestyle{empty}
\begin{landscape}
\subsubsection{Leltározás kezdeményezése}
Ha elérkezik a félévenkénti leltár ideje, a vezetőség egy tagja a rendszerbe belépve elindítja a leltár folyamatát. 
A rendszer ennek hatására befagyasztja az adatbázist, valamint értesítést küld a raktárosnak a leltár megkezdéséről. 
\vspace{5mm}

\noindent\textbf{Megjegyzés:} A leltározás fizikai eredményét (vagyis a raktárban ténylegesen tárolt alkatrészek számát típusok szerint) a raktáros viszi be a rendszerbe a leltár végén, amiről a vezetőség tagjai értesítést kapnak. 
Ekkor az a szoftver összehasonlítja a fizikai leltár eredményét az adatbázissal, és azt áttekinthető formában megjeleníti. 

\begin{figure}[!h]
    \centering
        \includegraphics[width=1.4\textwidth]{kepek/leltar_szekvencia.pdf}
        \caption{A leltározás kezdeményezésének szekvencia diagramja.}
\end{figure}
\end{landscape}

\thispagestyle{empty}
\begin{landscape}
\subsection{Design Class Diagram}
\begin{figure}[!h]
    \centering
        \includegraphics[width=1.3\textwidth]{kepek/dcd.pdf}
        \caption{A rendszer design class diagramja.}
\end{figure}
\end{landscape}
\end{document}