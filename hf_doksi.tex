\documentclass[11pt]{article}\usepackage[left=20mm,right=20mm,top=15mm,bottom=20mm]{geometry}

\usepackage[T1]{fontenc}
\usepackage[magyar]{babel}
\usepackage[utf8]{inputenc}
\usepackage{graphicx}
\usepackage{caption}
\usepackage{siunitx}
\usepackage{amsmath}
\usepackage{epstopdf}
\usepackage{multirow}
\usepackage{makeidx}
\usepackage[]{mcode}
\usepackage{placeins}
\usepackage{subcaption}
\usepackage{hyperref}

%\DeclareUnicodeCharacter{00A0}{~}

\begin{document}
\begin{titlepage}
\centering
	\includegraphics[width=0.5\textwidth]{figures/bme_logo_kicsi.eps}\par\vspace{1cm}
	\vspace{1cm}
	\vspace{1.5cm}
	{\huge\bfseries Szoftvertervezés 1. házi feladat \par}
	\vspace{15cm}
	{\huge\itshape Locskai Norbert, Kovács András, Barancsuk Lilla \par}
	\vfill

% Bottom of the page
	{\large \today\par}
\end{titlepage}

\section{Vízió}
\subsection{FEJLÉC!!!!!!!}

\subsection{Bevezetés}
A Service4U cég berendezések szervizelésével foglalkozik. Az Device4U nemzetközi cég által
gyártott berendezések szervizelését végzi Magyarországon. A Service4U cég az alkatrészeket
kizárólag a Device4U cégtől szerzi be.
A Service4U vezető egy olyan számítógépes rendszert szeretnének, mely segítségével képesek
lesznek a raktárban tárolt alkatrész készletet nyomon követni, a rendelés, felhasználás, szállítás, ill.
leltározás folyamatait támogatni.

\subsubsection{Megoldandó probléma}
Az alkatrészeket a Service4U cég raktáron tartja, arra törekszik, hogy minden alkatrészből legyen a
raktáron legalább két darab. Amikor az alkatrészeket a Service4U megrendeli, azokat mindig
azonos szállító cég szállítja szállítólevéllel. Minden szállításhoz egy szállítólevél tartozik és minden
szállításban egy adott rendelésből származó alkatrészek vannak. Az alkatrészek rendelését a cég
vezetői végzik.
Az alkatrészeket a Service4U cég szerelői vételezik ki a raktárból egy-egy berendezés javítása
során. A javításról minden esetben munkalap készül, mely tartalmazza a javításhoz felhasznált
alkatrészek listáját. A szerelő önállóan végzik egy-egy berendezés javítását.
Service4U cég vezető félévente leltározást végeznek a raktárban, amikor elkészítik azt a listát, ami a
raktárban levő alkatrészek listáját tartalmazza.
A Device4U viszonylak kevés különböző berendezést gyárt és azok évente változnak, azonos
alkatrészeket tartalmaznak, egy-egy berendezés sok alkatrészből állhat.

\subsection{Érdekeltek köre - stakeholderek + céljaik}
\begin{itemize}
\item[szerelő] :A szerelő az, aki végzi a gépek javítását. A rendszer segítségével szeretné meggyorsítani az elvégzett munka dokumentálását. (Mennyi alkatrészt használt fel.)
\item[] raktáros
\item[rendszergazda]
\item[vezető]: Ő viszi be az új alkatrészeket is.
\end{itemize}

\subsection{Rendszer részei}
\begin{itemize}
\item[] adatbázis: nyilvántartás
\item[] köztes rész: adatbázist és grafikus felületet összekötő interfész
\item[] GUI
\item[] azonosítófelület
\item[] lehetőség távoli hozzáférésre (helyi hálózat, internet)
\item[] vonalkódleolvasó - vonalkódfeldolgozó
\end{itemize}

\subsection{Rendszer korlátai, határai}
\subsubsection{Mit nem csinál?}
Nem automatizált a raktár feltöltése-ürítése, leltározás, raktáros csinálja.
Kollégák nyilvántartását nem veszi át automatikusan a céges rendszerből, kézzel kell bevinni az azonosítójukat.
Nem a rendszer feladata a rendelés elküldése, csak generálása.

\subsubsection{Technológiai korlátok}
Egy szerver van, mindenki böngészőn keresztül lép be...
(Autentikáció nem kell helyi gépről)


\section{Fogalomszótár - glossary}
\begin{itemize}
\item[berendezés]......
\item[alkatrész]
\item[szerelő]
\item[cégvezető]
\item[raktáros]
\item[szállítólevél]
\item[munkalap]
\item[rendszergazda]
\item[rendelés]
\item[leltározás]
\item[jelenéts]
\end{itemize}

\section{Kiegészítő követelmények leírása - supplementary specification}
\subsection{FEJLÉC}

\subsection{Bevezetés}
Röviden a rendszer célja, ezért kell, hogy a rendszer...

\subsection{Usability}
GUI egyszerű, ergonomikus

\subsection{Reliability}
Backlog az adatbázisban, rendszeres mentés, követhető, hogy
ki mikor mit csinált. Visszaállíthatóság. 
Munkanapokon éjfélkor készül egy biztonsági mentés,

\subsection{Performance}
Sok alkatrész: max 1000 féle alkatrészt tárol, max 1000 db mindegyikből. Rendszer max 500 féle gép alkatrészeit ismeri. Egy gép max 200 alkatrszéből állhat. Egyszerre max 10-en férhetnek hozzá a rendszerhez. 

\subsection{Supportability}
Adatbázis bővíthető legyen, grafikus interfész tetszőlegesen változtatható a használt böngésző korlátai között.
A cég vállalja a rendszeres karbantartást. (félévente randszeresen, igény szerint - telefonos support)

\subsection{Implementation constraints}
Legelterjedtebb böngészőkön (Safari, Chrome, Mozzilla, IE) működik.
Céges PC-k. (szar az összes... korlátozott proci, memóia) Céges szerver: ... (ez is szar)
Vonalkódleolvasó

\subsection{Interfaces}
Nem szükséges ilyennel együttműködni.

\section{Use Case modell}
\subsection{Aktorok}
\subsubsection{Elsődleges}
\subsubsection{Támogató}
\subsection{Háttér}

\subsection{User nyilvántartás}
rendszergazda

\subsection{Bejelentkezés}
raktáros, rendszergazda, vezetől

\subsection{Rendelés}
- azonosító : UC1
- aktorok: 
- stakeholderek:
- előfeltételek:
- hatása: postconditions
- basic flow:
- alternative flow:
- special requirements
- technology and data variations list.... 
- open issues
előfeltételek: ha 2-nél kevesebb van egy alkatrészből mindenképpen, de lehet enélkül is

\subsection{Automatikus értesítés}
vezetőknek, ha nincs elég egy alkatrészből

\subsection{Új gép bevitele}
vezetők

\subsection{Backlog lekérdezése}
vezetők

\subsection{Leltárazás}
- azonosító : UC2
- aktorok: A leltárazásban részt vesz a cégvezetés... Mit csinál?
- stakeholderek: Raktárosok számolják össze, hogy mennyi van...
- előfeltételek:
- hatása: postconditions
- basic flow:
- alternative flow:
- special requirements
- technology and data variations list.... 
- open issues
vezetők

\subsection{Megrendelés átvétele}
raktáros

\subsection{Alkatrész kiadása}
- azonosító : UC3
- aktorok: 
- stakeholderek:
- előfeltételek:
- hatása: postconditions
- basic flow:
- alternative flow:
- special requirements
- technology and data variations list.... 
- open issues

\subsection{Biztonsági mentés készítése}
timer csinálja

\subsection{Rendszer visszaállítása}
rendszergazda kezdeményezi

%
%\begin{figure}[!h]
%    \centering
%    \begin{subfigure}[b]{0.45\textwidth}
%        \includegraphics[width=\textwidth]{figures/1_5_3_ehkul.eps}
%        \subcaption{Torzítatlan eset.}
%    \end{subfigure}
%    \quad
%    \begin{subfigure}[b]{0.45\textwidth}
%        \includegraphics[width=\textwidth]{figures/5_feladat/ehkul.eps}
%        \subcaption{Torzított eset.}
%    \end{subfigure}
%    \caption{A lineáris kombinátor és a rendszer sorfejtett átviteli fügvényének együtthatóinak különbsége torzítatlan és torzított kimenetű rendszer esetén. Az együtthatók különbsége torzított esetben jelentős, hat nagyságrenddel nagyobb, mint torzítatlan esetben.}\label{fig:5_f_ehkul}
%\end{figure}


\FloatBarrier
\end{document}